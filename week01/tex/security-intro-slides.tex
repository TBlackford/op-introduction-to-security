% Beamer slide template prepared by Tom Clark <tom.clark@op.ac.nz>
% Otago Polytechnic
% Dec 2012

\documentclass[10pt]{beamer}
\usetheme{Dunedin}
\usepackage{graphicx}
\usepackage{colortbl}
	\usepackage{tabularx}
\newcommand\codeHighlight[1]{\textcolor[rgb]{1,0,0}{\textbf{#1}}}

\title{Measuring Risk}

\author[IN618]{Security}
\institute[Otago Polytechnic]{
  Otago Polytechnic \\
  Dunedin, New Zealand \\
}
\date{}
\begin{document}

%----------- titlepage ----------------------------------------------%
\begin{frame}[plain]
  \titlepage
\end{frame}


\begin{frame}
	\frametitle{How secure can we be?}
	
	More to the point, how secure can we \emph{afford} to be?
	
\end{frame}

\begin{frame}
	\frametitle{Security trade offs}

	You don't get security for free.  You always
	have to trade away something to get it.

	\begin{itemize}
		\item Money
		\item Time
		\item Convenience
		\item Capability
	\end{itemize}

	Often we give up some combination of all of these.
\end{frame}

\begin{frame}
	\frametitle{Are we getting a good deal?}

	\begin{itemize}
		\item Once we recognise that security has a cost,
			the question isn't really, ``How secure 
			can we be?''
		\item Instead, the question is, ``How much are
			we willing to trade away in return for 
			some security?''
		\item The thing is, people are not very good at 
			assessing their security trade-offs.
	\end{itemize}
\end{frame}

\begin{frame}
	\frametitle{Risk}

	\begin{itemize}
		\item In our daily lives we may be able to 
			tolerate our less than ideal decision making.
		\item In a business setting we need to do better.
		\item The name for this thing we need to measure is \emph{risk}.
		\item Once we know how much risk we are exposed to, we can make smart
		decisions about how to reduce it.
	\end{itemize}
\end{frame}

\begin{frame}
	\frametitle{Measuring risk}
	
	Risk is a factor of two things:
	\begin{itemize}
		\item The probability that something bad will happen
		\item The amount of harm that will result from it happening
	\end{itemize}
\end{frame}

\begin{frame}
	\frametitle{Aside: CIA triad}
	It's useful to identify the kinds of harm that can result from a security incident. 
	The \emph{CIA triad} is a helpful mnemonic for this.
	
	\begin{itemize}
		\item \textbf{C}onfidentiality
		\item \textbf{I}ntegrity
		\item \textbf{A}vailability
	\end{itemize}
\end{frame}

\begin{frame}
	\frametitle{Two types of analysis}
	
	\begin{itemize}
		\item Qualitative: We ranks our risks relative to each other and decide which ones receive priority.
		\item Quantitative: We assign a particular value, typically measured in dollars, to each element of risk.
	\end{itemize}
\end{frame}

\begin{frame}
	\frametitle{Qualitative Analysis}
	
	The goal of \emph{qualitative risk analysis} is to prioritise risks into categories like high, medium, and low priority.
\end{frame}

\begin{frame}
	\frametitle{Qualitative Analysis}
	
	To do this we evaluate
	
	\begin{itemize}
		\item The amount of harm caused by a possible event (high, medium, low)
		\item The relative probability of an event (high, medium, low)
	\end{itemize}
\end{frame}
	
\begin{frame}
	\frametitle{Qualitative Analysis}

	\begin{tabularx}{\textwidth}{|X|X|X|X|}
		\hline
		\begin{tabular}{ll} Probability $\rightarrow$ \\ Harm $\downarrow$ \end{tabular} &  Low & Medium & High \\ \hline
		Low & \cellcolor{green} & \cellcolor{green} & \cellcolor{yellow} \\ \hline
		Medium & \cellcolor{green}  & \cellcolor{yellow} & \cellcolor{red} \\  \hline
		High & \cellcolor{yellow} & \cellcolor{red} & \cellcolor{red} \\ \hline
	\end{tabularx}
\end{frame}
\begin{frame}
	\frametitle{Quantitative risk analysis}

	\begin{itemize}
		\item Another way to assess risk is to perform \emph{quantitative risk analysis}.
		\item This is more complicated than qualitative analysis, but it gives us a
		      way to set a budget for security improvement.
	\end{itemize}
\end{frame}

\begin{frame}
	\frametitle{Elements of risk analysis}

	To analyse our risk, we consider

	\begin{itemize}
		\item Assets - We assign a dollar value to them.
		\item Threats to those assets and the probability that
			the threatened harm will occur.
		\item Countermeasures that guard against the threatened harm
			or that reduce the amount of harm.  These
			have a cost that we measure in dollars.
	\end{itemize}
\end{frame}

\begin{frame}
	\frametitle{Am example}
	Suppose our business has a warehouse/shipping facility
	that ships orders to our customers at a rate of 
	\$1000 of revenue per hour.

	\begin{itemize}
		\item Asset:  There is a computer system that the staff
			use to process orders.
		\item Threat:  A power cut would take the system down.
		\item Countermeasure:  We could get a UPS and backup generator.
	\end{itemize}
\end{frame}

\begin{frame}
	\frametitle{Doing the numbers}

	\begin{itemize}
		\item The value of the asset is \$1000 per hour.
		\item Suppose we can expect 2 hours of power cuts in a typical year.
		\item Our risk (cost per hour of downtime) * (expected downtime per year)
			or \$1000 * 2 = \$2000
		\item This is called our \emph{annual loss expectancy}, or ALE.

	\end{itemize}
\end{frame}

\begin{frame}
	\frametitle{Doing the numbers}

	\begin{itemize}
		\item Now suppose we can install and operate
			a backup power system for \$6000.
		\item The system is expected to last for 5 years.
		\item Our annual cost is \$6000/5 = \$1200
		
	\end{itemize}
\end{frame}

\begin{frame}
	\frametitle{Doing the numbers}

	\begin{itemize}
		\item Now we can look at the cost/benefit.
		\item ((ALE without backup) - (ALE with backup))  \\ - (Annual cost of backup)
		\item  (\$2000 - 0) - \$1200 = \$800
        \item In other words, backup power reduces a \$2000 risk to 
			an \$800 risk. It's a good trade off.
	\end{itemize}
\end{frame}
\begin{frame}
	\frametitle{Another example}

	Suppose that your boss comes in on Monday morning after having heard 
	about cryptolocker-like malware attacks, in which an attacker encrypts
	all your files and then demands a ransom, say \$25,000 in return for the
	encryption key.  Your boss is in a near panic and insists on strong
	protective measures.
\end{frame}
\begin{frame}
	\frametitle{Determine the risk}

	\begin{itemize}
		\item Your research shows that the probability of 
			this malware intrusion event is only 
			0.0001 per week.
		\item Therefore, the probability this will happen in a given year is $0.0001 * 52 = 0.0052$
		\item This means that your risk exposure, or ALE is
			(cost of intrusion event) * (probability per week) * (52 weeks)
		\item ALE = \$25,000 * 0.0001 * 52 = \$130
		\item In other words the risk is very low, even though the cost and hence harm from an event is high.
	\end{itemize}
\end{frame}

\begin{frame}
	\frametitle{However...}

	Control measures for this risk include anti-malware software and
	backup/recovery systems.  So, the \$130 of risk exposure in this case
	can be added to a larger risk exposure suite when determining the budget 
	for malware protection and backup.
\end{frame}

\begin{frame}
	\frametitle{Conclusions}
	
	Qualitative analysis is, at least initially, fairly easy and is probably a good place to start.
	
	\vspace{10mm}
	Sometimes, especially when it comes to setting budgets, quantitative analysis provides more precise information and insight.
\end{frame}

\end{document}
