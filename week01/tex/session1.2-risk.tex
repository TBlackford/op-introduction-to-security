% Beamer slide template prepared by Tom Clark <tom.clark@op.ac.nz>
% Otago Polytechnic
% Dec 2012

\documentclass[10pt]{beamer}
\usetheme{Dunedin}
\usepackage{graphicx}
\usepackage{fancyvrb}

\newcommand\codeHighlight[1]{\textcolor[rgb]{1,0,0}{\textbf{#1}}}

\title{Managing Risk}

\author[IN618]{Security}
\institute[Otago Polytechnic]{
  Otago Polytechnic \\
  Dunedin, New Zealand \\
}
\date{}
\begin{document}

%----------- titlepage ----------------------------------------------%
\begin{frame}[plain]
  \titlepage
\end{frame}


\begin{frame}
	\frametitle{How secure can we be?}
	
	Last time we said that perfect security is not possible.
        Given that, we need a way to talks about how secure we are
        or can be.  

\end{frame}

\begin{frame}
	\frametitle{Security trade offs}

	You don't get security for free.  You always
	have to trade away something to get it.

	\begin{itemize}
		\item Money
		\item Time
		\item Convenience
		\item Capability
	\end{itemize}

	Often we give up some combination of all of these.
\end{frame}

\begin{frame}
	\frametitle{Are we getting a good deal?}

	\begin{itemize}
		\item Once we recognise that security has a cost,
			the question isn't really, ``How secure 
			can we be?''
		\item Instead, the question is, ``How much are
			we willing to trade away in return for 
			some security?''
		\item The thing is, people are not very good at 
			assessing their security trade-offs.
	\end{itemize}
\end{frame}

\begin{frame}
	\frametitle{Risk}

	\begin{itemize}
		\item In our daily lives we may be able to 
			tolerate our less than ideal decision making.
		\item In a business setting we need to do better.
		\item The name for this thing we need to measure is \emph{risk}.
		\item Risk has a standard unit of measure.  It's dollars per year.
	\end{itemize}
\end{frame}

\begin{frame}
	\frametitle{Quantitative risk analysis}

	\begin{itemize}
		\item One way to assess risk is to perform \emph{quantitative risk analysis}.
		\item Full risk analysis is the work of specialists, but we can
			perform some basic analysis on our own.
	\end{itemize}
\end{frame}

\begin{frame}
	\frametitle{Elements of risk analysis}

	To analyse our risk, we consider

	\begin{itemize}
		\item Assets - We assign a dollar value to them.
		\item Threats to those assets and the probability that
			the threatened harm will occur.
		\item Countermeasures that guard against the threatened harm
			or that reduce the amount of harm.  These
			have a cost that we measure in dollars.
	\end{itemize}
\end{frame}

\begin{frame}
	\frametitle{Am example}
	Suppose our business has a warehouse/shipping facility
	that ships orders to our customers at a rate of 
	\$1000 of revenue per hour.

	\begin{itemize}
		\item Asset:  There is a computer system that the staff
			use to process orders.
		\item Threat:  A power cut would take the system down.
		\item Countermeasure:  We could get a UPS and backup generator.
	\end{itemize}
\end{frame}

\begin{frame}
	\frametitle{Doing the numbers}

	\begin{itemize}
		\item The value of the asset is \$1000 per hour.
		\item Suppose we can expect 2 hours of power cuts in a typical year.
		\item Our risk (cost per hour of downtime) * (expected downtime per year)
			or \$1000 * 2 = \$2000
		\item This is called our \emph{annual loss expectancy}, or ALE.

	\end{itemize}
\end{frame}

\begin{frame}
	\frametitle{Doing the numbers}

	\begin{itemize}
		\item Now suppose we can install and operate
			a backup power system for \$6000.
		\item The system is expected to last for 5 years.
		\item Our annual cost is \$6000/5 = \$1200
		\item We call this the \emph{annual cost of control}, or ACC
	\end{itemize}
\end{frame}

\begin{frame}
	\frametitle{Doing the numbers}

	\begin{itemize}
		\item Now we can look at the cost/benefit.
		\item (ALE without backup) - (ALE with backup) - (Cost of backup)
		\item \$2000 - 0 - \$1200 = \$800
                \item In other words, backup power reduces a \$2000 risk to 
			an \$800 risk. It's a good trade off.
	\end{itemize}
\end{frame}
\begin{frame}
	\frametitle{Another example}

	Suppose that your boss comes in on Monday morning after having heard 
	about cryptolocker-like malware attacks, in which an attacker encrypts
	all your files and then demands a ransom, say \$25,000 in return for the
	encryption key.  Your boss is in a near panic and insists on strong
	protective measures.
\end{frame}
\begin{frame}
	\frametitle{Determine the risk}

	\begin{itemize}
		\item Your research shows that the probability of 
			this malware intrusion event is only 
			0.0001 per week.
		\item This means that your risk exposure, or ALE is
			(cost of intrusion event) * (probability per week) * (52 weeks)
		\item ALE = \$25,000 * 0.0001 * 52 = \$130
		\item In other words the risk is very low.
	\end{itemize}
\end{frame}

\begin{frame}
	\frametitle{However...}

	Control measures for this risk include anti-malware software and
	backup/recovery systems.  So, the \$130 of risk exposure in this case
	can be added to a larger risk exposure suite when determining the budget 
	for malware protection and backup.
\end{frame}

\end{document}
